\documentclass[11pt]{article}
%Increase the text height
\addtolength{\voffset}{-62pt}
\addtolength{\textheight}{62pt}
\usepackage{amsmath}

%Increase the text width
\addtolength{\hoffset}{-22pt}
\addtolength{\oddsidemargin}{-32pt}
\addtolength{\marginparsep}{-11pt}
\addtolength{\marginparwidth}{-45pt}
\addtolength{\textwidth}{110pt}
\usepackage{mathtools}
\begin{document}
\pagestyle{myheadings}
\markright{\sc Tunahan Yadigarbigun-Furkan Emre Yilmaz 230201003-230201057  %
\hfill  /p.}

\paragraph{Solution 1.}
\paragraph{a.} Using three point difference formula we are asked to approximate $f'(0)$, $f(x) = cos(x)$, $h = 0.01$
\paragraph{} Start with writing three point difference formula for second derivative aproximation with error term included:
\begin{eqnarray}
f''(x) &=& \frac{f(x-h) - 2f(x) + f(x+h)}{h^2} -  \underbrace{\frac{h^2}{12}f^\mathrm{\romannumeral 4}(c)\quad}_{\text{Error Term}} \
\end{eqnarray}
where we know that $c$ in the interval of $x-h < c < x+h$ and continue with substituting values to the formula (1) without error term included:
\begin{eqnarray*}
f''(0) &=& \frac{f(-0.01) - 2f(0) + f(0.01)}{(0.01)^2} \\
&=& \frac{cos(-0.01) - 2 cos(0) + cos(0.01)}{(0.01)^2} \\
&=& -0.999991666
\end{eqnarray*}
for c we can say $-0.01 < c < 0.01$ and error is
\begin{eqnarray*}
\frac{(0.01)^2}{12}cos(c)
\end{eqnarray*} 
\paragraph{b.} We are asked to find the error term for the approximation formula:
\begin{eqnarray*}
f'(x) &=& \frac{4f(x+h) - 3f(x) - f(x-2h)}{6h}
\end{eqnarray*}
\paragraph{}Using Taylor's Theorem we can make two expansion to reach our goal approximation:
\begin{eqnarray}
f(x+h) &=& f(x) + hf'(x) + \frac{h^2}{2}f''(x) + \frac{h^3}{6}f''' (c_1)\\
f(x-2h) &=& f(x) - 2hf'(x) + 2h^2f''(x) - \frac{4}{3}h^3f'''(c_2)
\end{eqnarray}
applying Generalized Intermediate Value Theorem we can combine $c_1=c_2=c$ and $x-2h < c < x+h$ and continue with multiplying (2) equation with 4:
\begin{eqnarray}
4f(x+h) = 4f(x) + 4hf'(x) + 2h^2f''(x) + \frac{2}{3} h^3f'''(c_1)
\end{eqnarray}
end substracting equation (4)(3) we obtain:
\begin{eqnarray*}
4f(x+h) &=& 4f(x) + 4hf'(x) + 2h^2f''(x) + \frac{2}{3} h^3f'''(c_1) \\
f(x-2h) &=& f(x) - 2hf'(x) + 2h^2f''(x) - \frac{4}{3}h^3f'''(c_2)
\end{eqnarray*}
result is:
\begin{eqnarray*}
4f(x+h) - f(x-2h) &=& 3f(x) + 6hf'(x) + 2h^3f'''(c) \\
6hf'(x) &=& 4f(x+h) - 3f(x) - f(x+2h) -2h^3f'''(c) \\
f'(x) &=& \frac{4f(x+h) - 3f(x) - f(x+2h)}{6h} -\underbrace{\frac{h^2}{3}f'''(c)}_{\text{Error Term}} \
\end{eqnarray*}
\paragraph{}We obtained the error term as
\begin{eqnarray*}
\frac{h^2}{3}f'''(c) 
\end{eqnarray*}
where $c$ in the interval of $x-2h < c < x+h$

\paragraph{Solution 2.a} For the composite Trapezoid Rule, $h$ is equal to $\frac{(b-a)}{m} $ and $a < c < b$
\begin{equation*}
\int_{a}^{b} f(x) dx = \frac{h}{2} (y_0 + y_m + 2 \sum\limits_{i=1}^{m-1} y_i )
 - \frac{(b-a)h^2}{12} f^{\prime\prime}(c)
\end{equation*}
\paragraph{} Taking $m=1$ and $h= \frac{\pi}{2}$ the approximation is 

\begin{equation*}
\int_{0}^{\frac{\pi}{2}} cosx dx \approx \frac{\pi}{4} (1 + 0 + 2[0] ) \approx 0,785
\end{equation*}
\paragraph{} The error is at most
\begin{equation*}
\frac{(\frac{\pi}{2})^3}{12} \left|-1\right| = 0.322982
\end{equation*}
\paragraph{} where the exact value is 1.

\paragraph{} Taking $m=2$ and $h= \frac{\pi}{4}$ the approximation is 

\begin{equation*}
\int_{0}^{\frac{\pi}{2}} cosx dx \approx \frac{\pi}{8} (1 + 0 + 2[0.7071] ) \approx 0,948
\end{equation*}
\paragraph{} The error is at most
\begin{equation*}
\frac{(\frac{\pi}{2}).(\frac{\pi}{4})^2}{12} \left|-1\right| = 0.080746
\end{equation*}
\paragraph{} where the exact value is 1.

\paragraph{} Taking $m=4$ and $h= \frac{\pi}{8}$ the approximation is 

\begin{equation*}
\int_{0}^{\frac{\pi}{2}} cosx dx \approx \frac{\pi}{16} (1 + 0 + 2[0.9238 + 0.7071 + 0.3826] ) \approx 0,987
\end{equation*}
\paragraph{} The error is at most
\begin{equation*}
\frac{(\frac{\pi}{2}).(\frac{\pi}{8})^2}{12} \left|-1\right| = 0.020187
\end{equation*}
\paragraph{} where the exact value is 1.

\paragraph{Solution 2.b} For the composite Simpson's Rule, $h$ is equal to $\frac{(b-a)}{2m} $ and $a < c < b$
\begin{equation*}
\int_{a}^{b} f(x) dx = \frac{h}{3} (y_0 + y_{2m} + 4 \sum\limits_{i=1}^{m} y_{2i-1} + 2 \sum\limits_{i=1}^{m-1} y_{2i})
 - \frac{(b-a)h^4}{180} f^{\prime\prime\prime\prime}(c)
\end{equation*}
\paragraph{} Taking $m=1$ and $h= \frac{\pi}{4}$ the approximation is 

\begin{equation*}
\int_{0}^{\frac{\pi}{2}} cosx dx \approx \frac{\pi}{12} (1 + 0 + 4[0.7071] + 2[0]) \approx 1.002273
\end{equation*}
\paragraph{} The error is at most
\begin{equation*}
\frac{(\frac{\pi}{2}).(\frac{\pi}{4})^4}{180} \left|-1\right| = 0.003321
\end{equation*}
\paragraph{} where the exact value is 1.

\paragraph{} Taking $m=2$ and $h= \frac{\pi}{8}$ the approximation is 

\begin{equation*}
\int_{0}^{\frac{\pi}{2}} cosx dx \approx \frac{\pi}{24} (1 + 0 + 4[0.9238 + 0.3826] + 2[0.7071]) \approx 1.000126
\end{equation*}
\paragraph{} The error is at most
\begin{equation*}
\frac{(\frac{\pi}{2}).(\frac{\pi}{8})^4}{180} \left|-1\right| = 0.0002075
\end{equation*}
\paragraph{} where the exact value is 1.

\paragraph{} Taking $m=4$ and $h= \frac{\pi}{16}$ the approximation is 

\begin{equation*}
\int_{0}^{\frac{\pi}{2}} cosx dx \approx \frac{\pi}{48} (1 + 0 + 4[0.9807 + 0.8314 + 0.5555 + 0.1950] + 2[0.9238 + 0.7071 + 0.3826]) \approx 1.0000081
\end{equation*}
\paragraph{} The error is at most
\begin{equation*}
\frac{(\frac{\pi}{2}).(\frac{\pi}{16})^4}{180} \left|-1\right| = 0.000013
\end{equation*}
\paragraph{} where the exact value is 1.
\newpage
\paragraph{Solution 2.c} Approximating the integral using Trapezoid Rule with 
\paragraph{} m = 32   results 6.89456
\paragraph{} m = 64   results 6.88927
\paragraph{} m = 128 results 6.88795

\paragraph{Solution 2.d} Approximating the integral using Simpson's Rule with 
\paragraph{} m = 32   results 6.88750936
\paragraph{} m = 64   results 6.88751044
\paragraph{} m = 128  results 6.88751058

\end{document}